\documentclass[12pt]{article}
\title{Invisibility in "The Third and Final Continent"}
\date{}
\usepackage{setspace}
\usepackage{csquotes}
\usepackage[margin=1in]{geometry}
\usepackage{fancyhdr}
\fancypagestyle{firststyle}
{
  \fancyhf{}
  \renewcommand{\headrulewidth}{0pt}
  \fancyhead[L]{S M Farhan Samir\\ ENG140\\ Evangeline \\January 30th\\Prompt 3}
}
\begin{document}
\maketitle
\doublespacing

\thispagestyle{firststyle}

Lahiri's stories in \textit{Interpreter of Maladies} have a peculiarly intensive concern with the day-to-day mundanities and slight idiosyncrasies of first and second-generation, south-asian immigrants. This concern was particularly surprising considering that there ought to be more pressing concerns in the life of a foreigner and yet Lahiri pays little attention to such matters.
As I progressed through the collection, the motivation behind this obsession was elucidated: all the discomfort, alienation, or acceptance that a foreigner may feel is articulated far more poignantly through the banal moments of an average day. 
In "A Temporary Matter", Lahiri depicts the deterioration of a relationship not through habitual arguments but rather the breaking of quaint traditions. In "A Real Durwan", she portrays alienation not in explicit acts of exclusion but rather the inability to sleep comfortably. Her obsession with the day-to-day mundanities of these characters is clarified in the final story of the collection: "The Third and Final Continent". 
In this final story, Lahiri criticizes the overlooking of community involvement and acceptance in integrating
an immigrant into a new society. Furthermore, she argues that being actively involved in the community requires efforts from both the immigrant and the existing community members. She makes her argument by juxtaposing the dark and dreary tone of narration while the narrator feels alienated and alone with the optimistic tone of narration when the narrator feels like a member of his community.\\

Within the first few months of arriving in Boston, the narrator of the Third and Final Continent spends much of his time on the outside looking in and it was clear that he felt alienated in this new, strange land. It was evident that he yearned to be seen; there's a particularly remarkable indication of this aforementioned desire when he witnessed an altercation between two strangers on the street: "She did not see me standing there, and eventually she continued on her way" (190). Her acknowledging him would be odd and unlikely in this scenario and yet he interjected a remark about her lack of acknowledgement anyway, as if even
a glance of acknowledgement would alleviate some of his loneliness. The validity of the interpretation of the previous quote is reinforced when the narrator is upset by his landlord's lack of sentimentality at his departure: "I did not expect any display of emotion, but I was disappointed all the same" (191). He was disappointed because the only person who he had cultivated any semblance of a relationship with in America was Mrs. Croft and yet there was nothing---no well-wishing nor invitations to visit---to show for it. Things were going relatively well for him on the outside---he managed to attain a respectable full-time job and find his own place to live---but the tone of the slight, day-to-day interactions of this novel indicate that something was amiss. \\

There is a complete tonal shift from the narrator's perspective once he's finally able to connect with an American. Before this
turning-point, the tone of the novel is dreary; phrases such as "I was not her son...I owed her nothing" (189), "I was not touched
by her words" (189) and "I ate cornflakes and milk, morning and night" (175) were commonplace. Furthermore, when he thinks about
the home that he left behind, he recalls not joyful memories with his family but rather the darkest moments of his life 
such as his parents' death. However, once the narrator visits a bedridden Mrs. Croft with his wife and they make friendly conversation and share laughs, the tone reverses entirely. It changes from utter dreariness to a quaint charm. The narrator
falls in love with his formerly estranged wife: "... and discovered pleasure and solace in each other's arms" (196). He
becomes well-acquainted with his community: "We discovered that a man named Bill sold fresh fish on Prospect Street, and that
a shop in Harvard called Cardullo's sold bay leaves and cloves" (196). It's not the finding of full-time employment
nor a place to live nor the actual change of location to a developed nation that brings the narrator satisfaction but rather
acceptance within that nation and the inner community that he belongs to. \\

Lahiri also underscores that acheiving this communal harmony is a two-way street---both the foreigner
and the existing community have to make respectable efforts at integration. The narrator had to reach out
and visit Mrs. Croft even though he had no reason to believe that she---a 103 year old woman---would want
anything to do with him outside of a tenant-landlord relationship. In fact, he was afraid that his wife's
strange attire would be revolting to her: "Mrs. Croft, who was still scrutinizing Mala from top to toe with what seemed to be placid disdain" (195). Mrs. Croft
also put concerted efforts into making the narrator comfortable which ought to have
been difficult considering her likely unfamiliarity with South Asian immigrants. Her calling the narrator's wife 
"perfect lady" (195) changed the narrator's perspective toward his wife---he perceived her
as a possible source of shame but Mrs. Croft enlightened him to a different view. \\

While the previous stories in the collection were written largely to provide intimate insight into the lives of
south-asian immigrants and their children, this story carries the pedagogical imperative that
was explained earlier. It's notable that this was the only story where the narrator is nameless.
I believe that this is because the story is meant to be parabolic in nature; that is, rather than
solely being an interesting narrative, it also conveys a universal truth. The truth in this parabola being 
that lack of community involvement necessarily equates with spiritual death. 
Immigrants are especially prone to such a demise given the hurdles that come with residing in a foreign land. \\

\begin{center}
\textbf{Works Cited} \\
Lahiri, Jhumpa. \textit{Interpreter of Maladies}. New York: Houghton Mifflin Harcourt Publishing Company, 2007.
\end{center}
\end{document}
