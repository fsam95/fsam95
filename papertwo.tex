\documentclass[12pt]{article}
\title{Paper Two}
\date{}
\usepackage{setspace}
\usepackage{csquotes}
\usepackage[margin=1in]{geometry}
\usepackage{fancyhdr}
\fancypagestyle{firststyle}
{
  \fancyhf{}
  \renewcommand{\headrulewidth}[0pt}
  \fancyhead[L]{S M Farhan Samir\\ ENG140\\ Evangeline \\Some date \\Prompt 13}
}
\begin{document}
\maketitle
\doublespacing

% Explore the representation of motherhood in one of the texts we've discussed. 
% You might want to think about whether this portrayal aligns with or perhaps
% challenges the traditional ways we tend to culturally construct this role. 
% Other questions you might consider are how this portrayal relates to issues 
% surrounding gender and/or labour, how the text's version of motherhood
% compares to representations of fatherhood, whether we are given more than
% one model for whatt a mother might be, and in what specific ways the mother/child
% relationship bears out some of the text's central thematic concerns. Is 
% "mother" a stable category in this text or is it being itnerrogated?
% Is the kind of motherhood we encounter familiar or strange? What does the
% text argue through this representation of motherhood?

% motherhood in Oscar Wao:
% first, you'd have to consider how motherhood is traditionally represented
\thispagestyle{firststyle}


\begin{center}
  \textbf{Works Cited} \\
Diaz, Junot, \textit{The Brief Wondrous Life of Oscar Wao}. New York: Riverhead Books, 2007.
\end{center}
\end{document}
