\documentclass[12pt]{article}
\title{Paper Two}
\date{}
\usepackage{setspace}
\usepackage{csquotes}
\usepackage[margin=1in]{geometry}
\usepackage{fancyhdr}
\fancypagestyle{firststyle}
{
  \fancyhf{}
  \renewcommand{\headrulewidth}{0pt}
  \fancyhead[L]{S M Farhan Samir\\ ENG140\\ Evangeline \\January 30th\\Prompt 3}
}
\begin{document}
\maketitle
\doublespacing

\thispagestyle{firststyle}

% TODO: Figure out how to format titles in an essay 
%       Make sure bibliography is correct
%       Read through handout to make sure formatting is correct
%       
%
Lahiri's stories in Interpreter of Maladies have a peculiarly [intensive/obsessive] concern with the day-to-day mundanities and slight idiosyncrasies of first and second-generation, south-asian immigrants. This concern was particularly surprising considering that there ought to be more pressing concerns in the life of a foreigner and yet Lahiri pays little attention to such matters.
As I progressed through the collection, the motivation behind this obsession was elucidated: all the discomfort, alienation, or acceptance that a foreigner may feel is articulated far more poignantly through the banal moments of an average day. 
In "A Temporary Matter", Lahiri depicts the deterioration of a relationship not through habitual arguments but rather the breaking of quaint traditions. In "A Real Durwan", she portrays alienation not in explicit acts of exclusion but rather the inability to sleep comfortably. Her obsession with the day-to-day mundanities of these characters is clarified in the final story of the collection: The Third and Final Continent. 
In this final story, Lahiri argues that successful immigration consists of the immigrant not solely being present but rather actively involved in their
community. I define "successful immigration" to mean that the immigrant is satisfied and motivated in their new home. Furthermore, Lahiri argues that being actively involved in the community requires efforts from both the immigrant and the
existing community members. She makes her argument in 3 parts: (1) juxtaposing the final story with earlier stories of 
less successful immigration; (2) illustrating that even though the narrator in "The Third and Final Continent" was ostensibly 
doing well for himself, he was deeply dissatisfied (3)  \\

%1st paragraph:
%Mrs Sen's failure to be accepted and the people not accepting her leads to her giving up

%2nd paragraph:
%Narrator being dissatisfied
%subsequent tonal shift after being accepted by Mrs. Croft
%Talk about how 
%String of subsequent successes 

%3rd paragraph
%j--------------
%jnarrator not having a name
%narrator talking in first person 
%	- Why is this significant? 
%Mention in your last paragraph how all this is argued with little things like being part of your community 

% She makes this argument in three parts within the final story: [by [making] a complete tonal shift in the narrator's perspective %when he finally feels a sense of membership in the [Boston?] community], .] 

Within the first few months of arriving in Boston, the narrator of the Third and Final Continent spends much of his time on the outside looking in and it was clear that he felt alienated in this new, strange land. It was evident that he yearned to be seen; there's a particularly remarkable indication of this aforementioned desire when he an altercation between two strangers on the street: "She did not see me standing there, and eventually she continued on her way" (190). Her acknowledging him would be odd and unlikely in this scenario and yet he interjected a remark about her lack of acknowledgement anyway, as if even
a glance of acknowledgement would alleviate some of his loneliness. The validity of the interpretation of the previous quote is reinforced when the narrator is upset by his landlord's lack of sentimentality at his departure: "I did not expect any display of emotion, but I was disappointed all the same" (191). He was disappointed because the only person who he had cultivated any semblance of a relationship with in America was Mrs. Croft and yet there was nothing---no well-wishing nor invitations to visit---to show for it. Things were going relatively well for him on the outside---he managed to attain a respected, full-time job and found
an apartment---but the tone of the slight, day-to-day interactions of this novel indicate that something was amiss. \\

There is a complete tonal shift from the narrator's perspective once he's finally able to connect with an American. Before this
turning-point, the tone of the novel is dreary; phrases such as "I was not her son...I owed her nothing" (189), "I was not touched
by her words" (189) and "I ate cornflakes and milk, morning and night" (175) were commonplace. When he thinks about
the home that he left behind, he recalls not joyful memories with his family but rather the darkest moments of his life 
such as his parents' death. However, once the narrator visits a bedridden Mrs. Croft with his wife and they make friendly conversation and share laughs, the tone reverses entirely. It changes from utter dreariness to a quaint charm. The narrator
falls in love with his formerly estranged wife: "... and discovered pleasure and solace in each other's arms" (196). He
becomes well acquianted with his community: "We discovered that a man named Bill sold fresh fish on Prospect Street, and that
a shop in Harvard called Cardullo's sold bay leaves and cloves" (196). It's not the finding of full-time employment
nor a place to live nor the actual change of location to a developed nation that brings the narrator satisfaction but rather
acceptance within that nation and the inner community that he belongs to.

Lahiri also underscores that acheiving this communal harmony is a [two-way street]---both the foreigner
and the existing community have to make respectable efforts at integration. The narrator had to reach out
and visit Mrs. Croft even though he had no reason to believe that she---a 103 year old woman---would want
anything to do with him outside of a tenant-landlord relationship. In fact, he was afraid that his wife's
strange attire would revolt her: "(INSERT QUOTE HERE BOIIIIIIIIIIIIIIIIIIIIIIIIIIIIII)". Mrs. Croft
also put concerted efforts into making the narrator comfortable. Her calling the narrator's wife 
"perfect" (page number?) changed the narrator's perspective toward his wife---he perceived her
as a possible source of shame but Mrs. Croft enlightened him to a different view. 

Lahiri masterfully articulates her point that 
By not providing a name for the narrator, she also (quietly) emphasizes that this point applies to not just this particular story but rather all and any foreigners.
The importance of communal integration is difficult to articulate since it's hard to understand without an
emphasis on small, intimate and ostensibly mundane moments of isolation and alienation. Thankfully, this is where Lahiri excels.



What's your overarching point? 
	- It wasn't
\begin{center}
  \textbf{Works Cited} \\
Lahiri, Jhumpa\textit{Interpreter of Maladies}. New York: Houghton Mifflin Harcourt Publishing Company, 2007.
\end{center}
\end{document}
