\documentclass[12pt]{article}
\title{Paper Two}
\date{}
\usepackage{setspace}
\usepackage{csquotes}
\usepackage[margin=1in]{geometry}
\usepackage{fancyhdr}
\fancypagestyle{firststyle}
{
  \fancyhf{}
  \renewcommand{\headrulewidth}{0pt}
  \fancyhead[L]{S M Farhan Samir\\ ENG140\\ Evangeline \\January 30th\\Prompt 3}
}
\begin{document}
\maketitle
\doublespacing

\thispagestyle{firststyle}

% TODO: Figure out how to format titles in an essay 
%       Make sure bibliography is correct
%       Read through handout to make sure formatting is correct
%       
%
Lahiri's stories in Interpreter of Maladies have a peculiarly [intensive/obsessive] concern with the day-to-day mundanities and slight idiosyncrasies of first and second-generation, south-asian immigrants. This concern was particularly surprising considering that there ought to be more pressing concerns in the life of a foreigner and yet Lahiri pays little attention to such matters.
As I progressed through the collection, the motivation behind this obsession was elucidated: all the discomfort, alienation, or acceptance that a foreigner may feel is articulated far more poignantly through the banal moments of an average day. 
In "A Temporary Matter", Lahiri depicts the deterioration of a relationship not through habitual arguments but rather the breaking of quaint traditions. In "A Real Durwan", she portrays alienation not in explicit acts of exclusion but rather the inability to sleep comfortably. Her obsession with the day-to-day mundanities of these characters is clarified in the final story of the collection: The Third and Final Continent. [Through this final story, she argues that the most important---and habitually overlooked---component of successfully integrating an immigrant is communal involvement and acceptance. She makes this argument in three parts within the final story: [by [making] a complete tonal shift in the narrator's perspective when he finally feels a sense of membership in the [Boston?] community], .] 

Within the first few months of arriving in Boston, the narrator of the Third and Final Continent spends much of his time on the outside looking in. It was evident that he yearned to be seen; there's a particularly remarkable indication of this aforementioned desire when he an altercation between two strangers on the street: "She did not see me standing there, and eventually she continued on her way" (190). Her acknowledging him would be odd and unlikely in this scenario and yet he interjected a remark about her lack of acknowledgement anyway. The validity of the interpretation of the previous quote is reinforced when the narrator is upset by his landlord's lack of sentimentality at his departure: "I did not expect any display of emotion, but I was disappointed all the same" (197). He was disappointed because the only person who he had cultivated any semblance of a relationship with was Ms. Croft.

\begin{center}
  \textbf{Works Cited} \\
Lahiri, Jhumpa\textit{Interpreter of Maladies}. New York: Houghton Mifflin Harcourt Publishing Company, 2007.
\end{center}
\end{document}
